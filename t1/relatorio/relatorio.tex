
\documentclass{article}

\usepackage{color}              % Controle das cores
\usepackage[utf8]{inputenc}
\usepackage[T1]{fontenc}
\usepackage[portuguese]{babel}
\usepackage[fleqn]{amsmath} % Permite diagramação matemática avançada
\PassOptionsToPackage{hyphens}{url}\usepackage{hyperref} % Permite adicionar url's clicáveis
\usepackage{amssymb}
\usepackage{authblk}

\DeclareMathOperator{\BigO}{O}

% Configura a exibição de urls
\hypersetup{
	colorlinks = true,
	urlcolor   = blue,
	linkcolor  = blue,
	citecolor  = red
}

\newenvironment{varalgorithm}[1]
{\algorithm\renewcommand{\thealgorithm}{#1}}
{\endalgorithm}



\title {Relatório de experimentos - Atividade 1 - Imagem integral e PGM}
\author{Felipe Constantino de Oliveira}
\affil{%
	Instituto de Matemática e Estatística - IME
	\par
	Universidade de São Paulo -- USP
}

\date{2018}
% ----
% Início do documento
% ----
\begin{document}
	
	% Retira espaço extra obsoleto entre as frases.
	\frenchspacing 
	
	% ----------------------------------------------------------
	% ELEMENTOS PRÉ-TEXTUAIS
	% ----------------------------------------------------------
	% \pretextual
	
	% ---
	% Capa
	% ---
	\maketitle
	% ---
	
	\section{Introdução}
	A proposta do trabalho é avaliar as seguintes implementações de processamento de imagens:
	
	Foi desenvolvido um programa em C para avaliação
	
	\begin{itemize}
		\item{} Implementado pela função \textit{getVarianceAccessingTwice}
		
		\item{} Implementado pela função \textit{getVarianceAccessingOnce}
		
		\item{} Implementado pela função \textit{getVarianceUsingIntegralImage}	
	\end{itemize}
	 
	
	\section{Resultados}
	Esta seção é para ver o que acontece com comandos de texto que definem
	
	
	\section{Conclusão}
	É possível perceber 	
	
	\section{Referências}
	Esta seção é para ver o que acontece com comandos de texto que definem
	
\end{document}
